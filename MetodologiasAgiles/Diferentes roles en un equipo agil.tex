\documentclass[12pt,a4paper]{article}
\usepackage[a4paper, total={150mm, 260mm}, left=35mm, right=20mm]{geometry}
\usepackage[utf8]{inputenc}
\usepackage[spanish]{babel}
\setlength{\parindent}{0pt}
\setlength{\parskip}{0.5em}

\author{Juan Murua Olalde}

\begin{document}

\subsection*{$\Rightarrow$ La `voz' del ``QUÉ''}
Una voz única (\begin{footnotesize}aunque es posible que esté transmitiendo la voz de varias personas detrás de ella\end{footnotesize}).
\\Dialoga con quien tenga que dialogar y va definiendo lo que se desea obtener.
\\Prioriza las tareas, para obtener en todo momento el máximo valor posible de lo que se va haciendo.
\\Es la responsable de los gastos; y la que decide qué se lanza o qué se para.
\\Tiene el derecho de recibir de la `tribu' estimaciones de lo que podrá costar lo que está solicitando obtener.

\subsection*{$\Rightarrow$ La `tribu' del ``CÓMO''}
Idealmente un grupo de no más de unas siete personas. Entre todas, tienen los conocimientos y habilidades suficientes para obtener lo que se desea obtener. Funcionan como un equipo autoorganizado.
\\Deciden cómo hacer el trabajo. Y lo hacen bien hecho, sin dejar flecos sueltos.
\\Tienen el derecho de solicitar aclaraciones en cualquier momento a la `voz' o a la `audiencia', para tener claro lo que se pretende conseguir.

\subsection*{$\Rightarrow$ El `comodín''}
Un facilitador neutral.
\\Ayuda a la `tribu' a no perder el rumbo y a la `voz' a no irse por las ramas.
\\ Anima a todos a ir mejorando continuamente.
\\No decide nada ni puede trabajar aisladamente del resto del equipo.
\\Es una persona honesta y realista. Comenta los temas con franqueza; pero siempre en la línea de mantener y mejorar la salud y el rendimiento del equipo.
\\Es un aflorador de problemas, para solucionarlos.
\\Es un removedor de obstáculos, para permitir a la `voz' y a la `tribu' trabajar a pleno rendimiento.

\subsection*{$\Rightarrow$ La 'audiencia'}
El grupo heterogéneo de todas las personas interesadas en el proyecto, (incluyendo a la `voz', la `tribu' y el `comodín').
\\Se preocupan por saber lo que se está haciendo. A intervalos regulares, se les presenta lo realizado. Ofrecen su opinión y su consejo.
\\Tienen el derecho a que toda la información sobre el proyecto esté fácilmente accesible.

\vspace{1cm}
Nota: Cuanto más estrechamente trabajen la `voz' y la `tribu' mejor para todos. Entre ambas ha de haber un diálogo fluido. No se trata tan solo de que la una pide cosas y la otra se las suministra de la mejor manera posible. Ambas han de funcionar como un equipo cohesionado que persigue la mayor generación posible de valor para el proyecto.

\begin{flushright}
es un resumen del capítulo\\``Scrum Roles: an Abstraction''\\del libro `The People's Scrum' de Tobias Mayer.
\end{flushright}

\end{document}